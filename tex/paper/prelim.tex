\section{Preliminaries}

We use boldface font for vectors. For example, \( \mathbf{a} = (a_1, a_2,
\ldots, a_n) \) and \( \mathbf{a}[i] \) denotes the \( i \)-th element of the
vector \( \mathbf{a} \). We take \( \uplus \) for the disjoint union of sets, \(
\mathcal{P}(A) \) for the power set of the set \( A \). \( \Sigma \) denotes an
alphabet, a finite set of symbols ranged over by \( a, b, c \), and \( \Sigma^*
\) denotes the set of finite words over \( \Sigma \), ranged over by \( w_1,
w_2, \ldots, w_n \). \( \epsilon \) denotes the empty string. We adopt 
the
standard convention of the literature~\cite{Thomas:1997,tata2007} and 
assume
familiarity with formal language theory~\cite{Kozen:1997}.

A finite automaton (\( \mathit{FA} \)) over an alphabet \( \Sigma \) is 
a 5-tuple \( \mathcal{M} = (Q, \Sigma, s, F, \delta) \) consisting of a 
finite set \( Q \) of states, disjoint from \( \Sigma \), ranged over by 
\( q, p, r \) or \( q_1, \ldots , q_n \), an initial state \( s \in Q 
\), a set of final states \( F \subseteq Q \), and a transition relation 
\( \delta \) on \( Q \times \Sigma \times Q \). We say the 
\(\mathit{FA}\) is deterministic if \( \delta \) relates any pairs in \( 
Q \times \Sigma \) to some unique state, otherwise non-deterministic.

\( \mathcal{B} = (\{0, 1\}, \bot, \perp, \neg, \wedge, \vee) \) is a boolean
algebra. Let \( Q = \{ q_0, q_1, \ldots, q_n \} \) be a set of states 
and we assume
that states are indexed. Here we also use \( Q \) as a set of propositional
variables and let \( q_i \in Q \) vary through \( \mathcal{B} \). A proposition
consists of constants, variables, and logical connectives. Let \( \mathit{Prop}
\) be a set of propositional formulas ranged over by \( \alpha, \beta, \gamma \)
and propositions are defined as follows: \(\alpha ::= q_i \in Q \mid 0 \mid 1
\mid \neg \alpha \mid \alpha \vee \beta \mid \alpha \wedge \beta\). We denote
subsets of \( \mathit{Prop} \) by \( \Gamma, \Pi, \Omega \). The truth
assignments of the proposition \(\mathcal{B}^Q\) is a set of mappings from each
state to boolean algebra. We use \( u, v, t : Q \rightarrow \mathcal{B} \) to
denote each truth assignment.

Since logical equivalence \( \Leftrightarrow \) is an equivalence relation over
\( \mathit{Prop} \), we can construct a quotient \(
\mathit{Prop}_{/\Leftrightarrow} \). If \( \alpha \) and \( \beta \) are
logically equivalent, then they are identical in \(
\mathit{Prop}_{/\Leftrightarrow} \) and belong to the class \( [\alpha] \). For
instance, a set of propositions \( \{ \bot, \neg q_0 \wedge q_0 \} \) is taken for
just the singleton \( \{ [\bot] \} \). Although \( \mathit{Prop} \), the set of
propositions, is infinite, we only consider the finite representative members \(
[\alpha] \in \mathit{Prop}_{/\Leftrightarrow} \). Thus we have \(
|\mathit{Prop}_{/\Leftrightarrow}| = 2^{2^{|Q|}} \). We implicitly refer to the
quotient \( \mathit{Prop}_{/\Leftrightarrow} \) as simply \( \mathit{Prop} \)
and to \( [\alpha] \) as \( \alpha \) respectively.  It follows that if \(
\alpha \Rightarrow \beta \) and \( \beta \Rightarrow \alpha \) then \( \alpha =
\beta \). Note the relation `\( \Rightarrow \)' is a partial ordering over \(
\mathit{Prop} \).

Under a truth assignment \( u \in \mathcal{B}^Q \), the interpretation \( I_u \)
maps a proposition to the boolean algebra, defined as follows: \( (\rm i) \) \(
I_u(c) = c \), where \( c \) is either \( \bot \) or \( \perp \), \( (\rm ii) \)
\( I_u(q_i) = u(q_i) \), and the interpretation for each logical connective is
defined as usual. We write \( u \models \alpha \) if \( I_u(\alpha) = 1 \) and
we abuse the notation to write \( u \models \{\alpha, \beta, .., \gamma \} \) if
\( u \models \alpha \vee \beta \vee ..\vee \gamma \). Note that \( u \not\models
\varnothing \) for any \( u \). By denoting \( \alpha[\beta/q_i] \), we
substitute all occurrences of \( q_i \) in \( \alpha \) with \( \beta \). \(
\alpha[\beta_i/q_i \ldots \beta_j/q_j] \) denotes the simultaneous 
substitution
of those propositional variables \( q_i, \ldots, q_j \).

Given \( \mathit{Var} \), a set of first-order variables ranged over by \( x_1,
x_2, \ldots, x_n \) and \( \mathit{Pred} \), a set of predicates ranged over by
\( P, Q, R \), let \( \Phi \) be a set of quantified formulas ranged over by \(
\varphi, \psi, \chi \). As shown below, the formula is either a literal or a
logical combination of \( \mathit{FOL} \)-formulas.
\[
  \varphi ::= P \mid \neg P
  \mid \varphi \vee \psi \mid \varphi \wedge \psi
  \mid \forall x_i .\ \varphi(x_i) \mid \exists x_i .\ \varphi(x_i) 
\]

The \( \mathit{FOL} \)-formula with free variables \( x_1, \ldots, x_n \) is
denoted by \( \varphi(x_1, \ldots, x_n) \). We assume the \( \mathit{FOL}
\)-formula is in negation normal form.
