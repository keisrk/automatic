\section{Introduction}

A SAT/validity checking problem of a quantified formula is decidable on
structures such as WS1S and FOL of Presburger Arithmetic. Those decidable
fragments of logics are widely used in various automated decision
procedures \cite{KlaEtAl:Mona}.

The structure enjoys decidability typically because it admits quantifier
elimination. So-called "automatic structures", on the other hand, have another
mechanism of deciding the quantified formulas on them \cite{}, where each element
in the carrier set has a string representation and each predicate has a
dedicated automaton which serves as an interpreter. Moreover the string
representation and automaton construction are devised in a way that the regular
set falls in the interpretation of the predicate. \todo{Relationship btw. the
  formula validity and the emptiness of regular set} We can perform set
operations over the regular sets along with the structure of an input formula so
that the resulting automaton recognizes the set of instances which satisfy the
formula. Hence language emptiness checking of the automaton yields decidable
decision procedure for the SAT/validity checking problem.

\todo{Difficulty of the problem, especially for NFA implementation of the
  regular set} Interpreting an existential quantifier sources nondeterminism to
the automaton while interpreting a negation requires the automaton to be
deterministic. The problem is known to have the non-elementary complexity due to
the determinization procedure which causes state explosion through the subset
construction. For finite automata, each state has a complex structure of nesting
sets and tuples due to the repetitive applications of subset and product
constructions. It also makes the acceptance condition counter-intuitive.

\todo{Existing works handle the problem?}
\todo{more in detail} Antichain WS1S \cite{Fiedor2015,Fiedor2017}
\todo{more in detail} Syntactical approach \cite{Traytel15,TraytelN15}
In the AFA setting, propositions symbolically represent reachable states and
their acceptance follows a truth valuation. The determinization is directly
expressed by a sequence of `or' symbols. Expressed by logical connectives, the
complex structure in DFA is handled in a unified manner. The antichain algorithm
which maintains the equisatisfiability is a practically efficient method to
prune the reachable states. AFA naturally allows the equisatisfiable
transformation similar to the antichain algorithm. For each sub-formula of a
proposition we replace it with another sub-formula in the set if they are ordered
by implication. This operation is conducted at the same time the construction
proceeds. As a result, instead of maintaining reachable states we only keep a
set of propositions pairwise incomparable by implication to judge the language
emptiness.

\todo{What's new in our work?}  To handle formula not limited to prefix normal
form, we extend the antichain algorithm from \cite{Wulf2006} to utilize
implication as the pruning criteria. We list our contributions below.

\begin{itemize}
\item Generalization of the antichain algorithm.
\item Bisimulation up to congruence technique \cite{BonchiP13} as well.
\item \todo{An experiment with a randomly generated Presburger formulas}
\item \todo{A performance improvement of the theorem proving technique}
\end{itemize}

And the results shown in this paper contains the following caveats.
\begin{itemize}
\item Our technique cannot handle WS1S formulas.\todo{examples}
\end{itemize}

The paper is organized as follows; Section 2 introduces preliminary definitions
and notations. Section 3 introduces the definition of the alternating automaton
and closure property and DFA conversion. Section 4 describes the antichain
algorithm. We report on the experience with a prototype implementation of the
algorithm in Section 5, and discuss related work in Section 6. We conclude the
paper in Section 7. Appendix contains the omitted proofs.
