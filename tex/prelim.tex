\section{Preliminaries}

We use boldface font for vectors. For example, \( \mathbf{a} = (a_1, a_2,
\cdots, a_n) \) and the \( i \)-th element of the vector \( \mathbf{a} \) is
denoted by \( \mathbf{a}[i] \). We take \( \uplus \) for the disjoint union of
sets, \( \mathcal{P}(A) \) for the powerset of the set \( A \). \( \Sigma \)
denotes an alphabet, a finite set of symbols ranged over by \( a, b, c \), and
\( \Sigma^* \) denotes the set of finite words over \( \Sigma \), ranged over by
\( w_1, w_2, \cdots, w_n \). \( \epsilon \) denotes the empty string. We adopt
the standard convention of the literature\cite{Thomas:1997,tata2007} and assume
familiarity with formal language theory\cite{Kozen:1997}.

A finite automaton(\( \mathit{FA} \)) over an alphabet \( \Sigma \) is a 5-tuple
\( \mathcal{M} = (Q, \Sigma, s, F, \delta) \) consisting of a finite set \( Q \)
of states, disjoint from \( \Sigma \), ranged over by \( q, p, r \) or
\( q_1, \cdots , q_n \), an initial state \( s \in Q \), a set of final states
\( F \subseteq Q \), and a transition relation \( \delta \) on \( Q \times
\Sigma \times Q \). We say the \(\mathit{FA}\) is deterministic if \( \delta \)
relates any pairs in \( Q \times \Sigma \) to some unique state, otherwise
non-deterministic.

\( \mathcal{B} = (\{0, 1\}, 0, 1, \neg, \wedge, \vee) \) is a boolean
algebra. We let \( \mathit{Prop} \) denote the set of propositional formulas
built from logical connectives and states, i.e., \( \mathit{Prop} = T(\{ 0, 1,
\neg, \wedge, \vee \}, Q) \). Let \( Q = \{ q_0, q_1, ..., q_n \} \) be a set of
states and we assume that states are indexed.  Here we also use \( Q \) as a set
of propositional variables and let \( q_i \in Q \) vary through \( \mathcal{B}
\). A proposition consists of constants, variables, and logical connectives. Let
\( \mathit{Prop} \) be a set of propositional formulas ranged over by \( \alpha, \beta,
\gamma \).

\[
\alpha ::= q_i \in Q \mid 0 \mid 1
       \mid \neg \alpha \mid \alpha \vee \beta \mid \alpha \wedge \beta
\]

\( u : Q \rightarrow \mathcal{B} \) is a mapping from a state to boolean
algebra. \(\mathcal{B}^Q\) is a set of such mappings. For the truth assignments
of the proposition, we use \( u, v, t \).

Since logical equivalence \( \Leftrightarrow \) is an equivalence relation over
\( \mathit{Prop} \), we can construct a quotient \(
\mathit{Prop}_{/\Leftrightarrow} \). If \( \alpha \) and \( \beta \) are
logically equivalent, then they are identical in \(
\mathit{Prop}_{/\Leftrightarrow} \) and belong to the class \( [\alpha] \). For
instance, a set of propositions \( \{ 0, \neg q_0 \wedge q_0 \} \) is taken for
just the singleton \( \{ [0] \} \). Although \( \mathit{Prop} \), the set of
propositions, is infinite, we only consider the finite representative members \(
[\alpha] \in \mathit{Prop}_{/\Leftrightarrow} \). Thus we have \(
|\mathit{Prop}_{/\Leftrightarrow}| = 2^{2^{|Q|}} \). We implicitly refer to the
quotient \( \mathit{Prop}_{/\Leftrightarrow} \) as simply \( \mathit{Prop} \)
and to \( [\alpha] \) as \( \alpha \) respectively.  It follows that if \(
\alpha \Rightarrow \beta \) and \( \beta \Rightarrow \alpha \) then \( \alpha =
\beta \). Note the relation `\( \Rightarrow \)' is a partial ordering over \(
\mathit{Prop} \).

Under truth assignment \( u \), the interpretation \( I_u \) maps a proposition
to the boolean algebra, defined as follows: \( (\rm i) \) \( I_u(c) = c\), where
\( c \) is either \( 0 \) or \( 1 \), \( (\rm ii) \) \( I_u(q_i) = u(q_i) \),
and the interpretation for each logical connective is defined as usual. We write
\( u \models \alpha \) if \( I_u(\alpha) = 1 \) and we abuse the notation to
write \( u \models \{\alpha, \beta, .., \gamma \} \) if \( u \models \alpha \vee
\beta \vee ..\vee \gamma \). Note that \( u \not\models \varnothing \) for any
\( u \). By denoting \( \alpha[\beta/q_i] \), we substitute all occurrences of
\( q_i \) in \( \alpha \) with \( \beta \). We also write \( \alpha[\beta_i/q_i
  \ ...\ \beta_j/q_j] \) to denote the simultaneous substitution of those
propositional variables \( q_i, ..., q_j \). We denote subsets of \(
\mathit{Prop} \) by \( \Gamma, \Pi, \Omega \).

Given \( \mathit{Var} \), a set of first-order variables ranged over by \( x_1,
x_2, \cdots, x_n \) and \( \mathit{Pred} \), a set of predicates ranged over by
\( P, Q, R \), let \( \Phi \) be a set of quantified formulas ranged over by \(
\varphi, \psi, \chi \). We assume the \( \mathit{FOL} \)-formula is in negation
normal form. As shown below, the formula is either a literal or a logical
combination of \( \mathit{FOL} \)-formulas.
\[
  \varphi ::= P \mid \neg P
  \mid \varphi \vee \psi \mid \varphi \wedge \psi
  \mid \forall x_i .\ \varphi \mid \exists x_i .\ \varphi 
\]

